\chapter{Conclusions}\label{ch6}
\setlength{\epigraphrule}{0pt}
\setlength{\epigraphwidth}{0.75\textwidth}
\epigraph{\textit{The true delight is in the finding out rather than in the knowing.}}{Isaac Asimov}


In this work we have introduced the \acrlong{v2dm} to solve the many-body problem.
The \acrfull{2dm} contains all necessary information to describe such a system, and the expectation value of one- or two-particle operators
can be expressed as a linear function of the \gls{2dm}.
Unlike the more conventional quantum mechanical methods, the wave function is never used. This method itself has a long history and attracted quite some attention in the second half of
the previous century. At first glance, it has many interesting properties: the \gls{2dm} has a much better scaling than the wave function,
and the method is strictly variational. Unlike wavefunction-based methods, it produces a strict lower bound on the energy (instead of an
upper bound).
Unfortunately, the complexity of the many-body problem has not disappeared, but is shifted to the $N$-representability problem: what are the
necessary and sufficient conditions for a \gls{2dm} to be derivable from an ensemble of many-fermion wave functions?
In the 1960's, there was still hope that this problem could be solved in some way, but time has learned that it is a very hard problem (see
later).

% vim: spell spelllang=en syntax=tex tw=140 
