\chapter{Samenvatting}\label{dutch-summary}

\selectlanguage{dutch}
\hyphenation{re-pres-en-teer-baar-heid}

Alle materie is opgebouwd uit atomen. Democritus had dit al in de 5$^\text{de}$ eeuw v.Chr. gepostuleerd, maar hij kon dit natuurlijk
niet bewijzen. De wereld van het atoom bleek echter moeilijk te doorgronden. Pas vanaf de 19$^\text{de}$ eeuw kwam er echt schot in de zaak.
Men ontdekte dat atomen niet ondeelbaar waren en vond het deeltje dat wij tegenwoordig kennen als het elektron. 
In het begin van de 20$^\text{ste}$ eeuw schakelde de ontdekkingstocht een versnelling hoger.

\selectlanguage{english}

% vim: spell spelllang=nl syntax=tex tw=140 
